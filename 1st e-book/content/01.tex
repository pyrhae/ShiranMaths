\section{排列组合の战略}
    初学者在不一一列举的情况下,很难直观地想清楚哪些算重了,哪些算漏了。可以称之为,玄学问题。

    学习的关键在于:你得非常明确一些基本模型,这些基本模型往往只用很小的数字就能说明,想清楚后再做一些数字大的问题。

\section{排列组合の基本知识点}
    排列数公式(把n样东西放在m个位置位置): $A_n^m = \frac{n!}{(n-m)!}=n(n-1)(n-2)\ldots (n-m+1)$
    \\

    虽然书上每次讲到这个公式时一般以阶乘的形式给出,但实际计算中,往往不用阶乘。记法:从大的数开始乘,乘“小的数字那么多”个。
    \\

    组合数公式(从n样东西中拿m个): $C_n^m = \frac{n!}{(n-m)!m!} = \frac{n(n-1)(n-2) \ldots (n-m+1)}{m(m-1)(m-2) \ldots 1}$
    \\

    组合数公式就是在排列数公式上除以一个 $m!$。但实际计算中往往不用阶乘。记法:从大的数开始往小乘,乘“小的数字那么多”个,再除以“小的数字开始往小乘,乘小的数字那么多个”。

    \subsection{理解两个基本公式}
    排列数公式我们可以通过分步乘法计数原理去理解。
    \\

    理解组合数公式。我们考虑从 $n$ 个人取 $m$个人出来,不排队,不在乎顺序,即 $C_n^m$。如果在乎顺序,就是$A_n^m$,如果不在乎顺序,就要除掉重复,那么重复了多少?同样选出来的 m 个人,他们还要 “全排”得 $A_n^m$,所以 $C_n^m \cdot m! = A_n^m$。从中得到组合数公式。
    \\

    排列总是和组合连在一起,组合就是排列的一种特殊情况,组合就是排列不考虑顺序的一种计数方法。既然这样,我们计算组合数可以先计算排列数,然后除以重复度,就得到组合数。组合数公式分母中的$m!$就是重复度,就是前面置换的排列数。比如$AB$和$BA$这两种在组合里算一种,在排列里算$2!$种。

    \subsection{组合相关公式}
    \begin{enumerate}
        \item $C_n^m = C_n^{n-m}$  (对称性) \\ 做值日问题:四个同学中,选三名同学做值日就相当于选一名同学放学直接回家。该公式对于运算 $C_{10}^8$这样得组合时非常有用,直接转化成$C_{10}^2$。
        \item $C_n^{m-1}+C_n^m = C_{n+1}^m$  (组合数的递推式(杨辉三角的公式表达)) \\  记法:上面的数字取大的,底下的数字加一。
        \item $C_n^0 + C_n^1 + \ldots + C_n^n = 2^n$ \\ 抓兔子问题
    \end{enumerate}

\newpage
\section{排列组合の题型}

