\usepackage[top=2.3cm,bottom=1.5cm,
left=1.7cm,right=1.3cm,headheight=40pt]{geometry}

\usepackage{amsmath,amssymb,amsthm}

\usepackage[dvipsnames,x11names]{xcolor}
\usepackage[skins,theorems]{tcolorbox}
\usepackage{mathtools,nccmath,tabu}
% \usepackage{mathspec}
% \usepackage[BoldFont,SlantFont]{xeCJK}
\usepackage{tikz}
\usepackage{tikz-qtree}
\usepackage{tkz-euclide}
\usetikzlibrary{trees,intersections}
\usetikzlibrary{calc,arrows.meta,matrix,
	decorations.pathmorphing,decorations.pathreplacing,
	backgrounds,shadings,fit,shadows,positioning}
%\usetkzobj{all}

\usepackage[Symbol]{upgreek}
\usepackage{pgf,txfonts,wrapfig,wasysym,setspace}
\usepackage{fancybox,graphicx,enumerate,enumitem,systeme,arcs}

\usepackage{CJKfntef,ulem,fancyhdr,overpic,tabu,nicefrac,subfig}
\usepackage{polynom,longdivision}
\usepackage[loose]{units}
\usepackage{indentfirst}
\setlength{\parindent}{2em}

\usepackage[explicit]{titlesec}
\usepackage[dotinlabels]{titletoc}

\usepackage{pdfpages}

\usepackage[T1]{fontenc}
\usepackage{fontspec}
% \setCJKmainfont{Noto Serif CJK SC}
% \setCJKfamilyfont{f03}[Scale=0.9545]{Noto Sans Mono CJK SC Regular}

\setmainfont[Scale=0.9545]{LibreBaskerville-Regular}
\newfontfamily\fit[Scale=.92]{Georgia Italic}
\newfontfamily\fb{Junicode-Bold}
%\newfontfamily\fc[Scale=1.24]{EB Garamond}
\newfontfamily\fd[Scale=1.03]{Book Antiqua}
%\newfontfamily\fe{Bodoni 72 Oldstyle}
%\newfontfamily\fsec{Helvetica Neue}
%\newfontfamily\Arial[Scale=1.01]{Arial}
\newfontfamily\myGreek{Times New Roman}

\usepackage{fourier}

\allowdisplaybreaks

\newcommand{\hl}[1]{\bgroup\markoverwith
	{\textcolor{yellow!80!}
		{\rule[-.5ex]{2pt}{2.5ex}}}\ULon{#1}}
\delimitershortfall-1sp
\newcommand{\abs}[1]{\left| #1 \right|}
\newcommand{\sfrac}[2]
{\text{\scalebox{1.28}[1.24]{\(\frac{#1}{#2}\)}}}
\newcommand{\ifrac}[2]
{\text{\scalebox{1.4}[1.3]{\(\frac{#1}{#2}\)}}}
\newcommand{\iifrac}[2]
{\text{\scalebox{1.5}[1.42]{\(\frac{#1}{#2}\)}}}


\newcommand{\s}{\hspace{0.1em}}

\newcommand{\sol}{\par\tikz {\node
		[fill=gray!30!white,
		opacity=.8,
		rounded corners=4pt]
		{~解};}\par\medskip}

\newcommand{\circled}[2][]{\tikz[baseline=(char.base)]
	{\node[shape = circle, draw, inner sep = 1pt]
		(char) {\phantom{\ifblank{ #1}{#2}{ #1}}};%
		\node at (char.center) {\makebox[0pt][c]{#2}};}
		}
\robustify{\circled}


\DeclareMathOperator{\diff}{d\!}
\DeclareMathOperator{\cosec}{cosec}


\pagestyle{fancy}
\fancyfoot{}
\fancyfoot[RO,LE]{\thepage}
\fancyfoot[RE,LO]{牧羊教育}
\fancyhead[LE,RO]{
\includegraphics[scale=.43]{images/logo.jpg}}


\renewcommand{\headrulewidth}{0.6pt}

\newcommand{\ve}[1]{\overrightarrow{#1}}

\newcommand{\mlim}[1]{\medop\lim\limits_{#1}}
\newcommand{\msum}[2]{\medop\sum\limits_{#1}^{#2}}

\newcommand{\hchapter}{.755\textheight}

\def\IR{\varmathbb{R}}
\def\IN{\varmathbb{N}}
\def\IZ{\varmathbb{Z}}
\def\IQ{\varmathbb{Q}}
\def\IC{\varmathbb{C}}

\def\dx{\diff x}
\def\dy{\diff y}
\def\dt{\diff t}
\def\dz{\diff z}
\def\ds{\diff s}
\def\dS{\diff S}
\def\dA{\diff A}
\def\du{\diff u}
\def\dv{\diff v}
\def\dV{\diff V}
\def\dr{\diff r}
\def\dtheta{\diff\theta}

\def\dopx{\frac{\diff}{\diff x}}
\def\dopm{\mfrac{\diff}{\diff x}}
\def\dopy{\frac{\diff}{\diff y}}
\def\dopym{\mfrac{\diff}{\diff y}}
\def\dopt{\frac{\diff}{\diff t}}
\def\doptm{\mfrac{\diff}{\diff t}}
\def\dydx{\frac{\diff y}{\diff x}}
\def\dydxm{\mfrac{\diff y}{\diff x}}
\def\dxdy{\frac{\diff x}{\diff y}}
\def\dxdym{\mfrac{\diff x}{\diff y}}
\def\dydt{\frac{\diff y}{\diff t}}
\def\dydtm{\mfrac{\diff y}{\diff t}}
\def\dxdt{\frac{\diff x}{\diff t}}
\def\dxdtm{\mfrac{\diff x}{\diff t}}
\def\dydxx{\frac{\diff\!^2 y}{\dx^2}}

\newcommand{\ol}[1]{\overline{#1}}
\newcommand{\Exp}{\mathrm{E}}
\newcommand{\Var}{\mathrm{Var}}
\def\rP{\mathrm{P}}
\def\rC{\mathrm{C}}
\def\rX{\mathrm{X}}
\def\rY{\mathrm{Y}}
\def\rZ{\mathrm{Z}}
\def\rPi{\mathrm{\pi}}
\def\card{\mathrm{card}}
\def\bx{{\mathbf x}}
\def\bX{{\mathbf X}}
\def\bf{{\mathbf f}}
\def\bF{{\mathbf F}}
\def\br{{\mathbf r}}
\def\bi{{\mathbf i}}
\def\bj{{\mathbf j}}
\def\bk{{\mathbf k}}
\def\bs{{\mathbf s}}
\def\bv{{\mathbf v}}
\def\ba{{\mathbf a}}
\def\bb{{\mathbf b}}
\def\bc{{\mathbf c}}
\def\bF{{\mathbf F}}
\def\bG{{\mathbf G}}
\def\bP{{\mathbf P}}
\def\bA{{\mathbf A}}
\def\bB{{\mathbf B}}
\def\bC{{\mathbf C}}
\def\bD{{\mathbf D}}
\def\bM{{\mathbf M}}
\def\bS{{\mathbf S}}
\def\bT{{\mathbf T}}
\def\bU{{\mathbf U}}
\def\bQ{{\mathbf Q}}
\def\bzero{{\mathbf 0}}
\def\bI{{\mathbf I}}
\def\dbr{\diff\br}
\def\degree{^{\circ}}

\newcommand{\cvec}[2]{\Big(\,\begin{mmatrix}
#1\\[-1mm]#2
\end{mmatrix}\,\Big)}
\newcommand{\cvect}[3]{\Bigg(\,\begin{mmatrix}
#1\\[-1mm]#2\\[-1mm]#3
\end{mmatrix}\,\Bigg)}

\linespread{1.3}


\tikzstyle{every picture}+=[remember picture]

\tikzstyle{notesty}
=[draw=red, %fill=pink!10,
very thin,rectangle,
rounded corners, inner sep=10pt, inner ysep=16pt,
text=black]
\tikzstyle{notestytitle}
=[fill=red!90, text=white]


\newtcolorbox{Note}[1]{enhanced,title={\fb note },
		colframe=red,colback=white,
		arc=2pt,colbacktitle=red!80,
		coltitle=white,
		boxrule=.35pt,
		width={#1},
		before skip=7pt, after skip=15pt,
		drop lifted shadow=red!70,
		attach boxed title to top left=
		{xshift=1.6mm,yshift=-2.2mm},
		boxed title style={enhanced,
			skin=enhancedfirst jigsaw,
			size=small,arc=0pt,
			interior style={top color=red!60,
				bottom color=red!80}}}

\newcounter{Ex}[chapter]
\renewcommand{\theEx}{\arabic{Ex}}
\newcommand{\MyExample}
{{\CJKfamily{f12}范例} \refstepcounter{Ex}\scalebox{1.02}[1.02]
	{\textbf{\fd\theEx}}\hspace{1pt}}

\tcbuselibrary{skins,breakable,xparse}

\newtcolorbox{basicproperty}[1][\textheight]{
	enhanced,
	colframe=%cyan!74,
	black!70,
	colback=white,
	leftrule=1pt,
	rightrule=1pt,
	bottomrule=1pt,
	boxsep=1mm,
	top=4mm,bottom=3mm,
rounded corners,
arc=2.5mm,
outer arc=1mm,
arc is curved,
borderline={0.65pt}{2mm}{gray},
%borderline={1pt}{-3pt}{blue},
width=\textwidth,
height=#1,
breakable
}

\newtcolorbox{exercise}{
	colframe=%cyan!74,
	black!70,
%	fonttitle=\bfseries,
	leftrule=2.1pt,
	rightrule=2.1pt,
	bottomrule=2pt,
	boxsep=1mm,
	top=4mm,bottom=3mm,
	opacityback=0,
	rounded corners,arc=4pt,
	width=\textwidth,
	enhanced jigsaw,
	before skip=5mm,
	after skip=5mm,
	breakable,
	title=\MyExample
}

\newcommand{\solu}[1]{\begin{tcolorbox}[enhanced,colback=white,boxrule=0pt,frame hidden,
		borderline west={1.3pt}{-1mm}{black!65!},breakable]
		\sol #1
\end{tcolorbox}}

\newcommand{\blank}[1]{\begin{tcolorbox}[enhanced,colback=white,boxrule=0pt,frame hidden,
		borderline west={1.3pt}{-1mm}{black!65!}]
		\sol\vspace{#1}
\end{tcolorbox}}

\newcommand{\mycases}[2]
{\left\{\vphantom{\begin{array}{c} a\\[#1]
	\end{array}}\right.\kern#2}

\newtcbox{\tcbeqnote}[1][gray]{on line,
	arc=3pt,outer arc=0pt,colback=#1!8!white,colframe=#1!70!black,
	boxsep=0pt,left=1pt,right=1pt,top=2pt,bottom=2pt,
	boxrule=0pt,bottomrule=.5pt,toprule=.5pt,
	leftrule=.5pt,rightrule=.5pt}

\fboxrule=.8pt

\newcommand{\eqnote}[1]{\tcbeqnote{\footnotesize\s#1\s}}


\newcommand{\eqnotec}[2]{\tcbeqnote[#1]{\footnotesize\s#2\s}}

\newcommand{\Eqnote}[1]{
	\fcolorbox{miunblue}{miunblue!8!white!80}{ #1}}

\newcommand{\solnote}[2]{\vspace{-8mm}\begin{flushright}
	\begin{tcolorbox}
		[arc=6pt,outer arc=0pt,width={#1},
		boxsep=0pt,boxrule=.9pt,left=2mm,right=0pt]
		\small #2
	\end{tcolorbox}
\end{flushright}}

\definecolor{emphcol}{RGB}{175,175,175}

\newtcbox{\emphbox}[1][red]{on line,
	arc=0pt,
	outer arc=0pt,
	colback=#1!4!white,
	colframe=#1!70!black,
	boxsep=0pt,
	left=1pt,right=1pt,
	top=2pt,bottom=2pt,
	boxrule=0pt,
	bottomrule=1pt,
%	toprule=.6pt
}




\definecolor{miunblue}{RGB}{20,120,199}
\definecolor{miunyellow}{RGB}{252,210,19}


\colorlet{markc1}{blue!21!white!70}
\colorlet{markc2}{red!24!white!70}
\colorlet{markc3}{miunyellow!65!white!70}
\colorlet{markc4}{miunblue!45!white!70}


\newcommand{\tikzmark}[2]
{\tikz[baseline]{\node[anchor=base,inner sep=1.2pt] (#1)
		{#2};}}

\newcommand{\tikzmarkfill}[3]
{\tikz[baseline]{\node[fill=#3,anchor=base,inner sep=1.2pt] (#1)
		{#2};}}
	

\newcommand{\Exercise}{\begin{center}\medskip
\tikz[baseline]{\node[anchor=base] (exercise)
	{{\fit \huge Exercise} };}
\end{center}
	\begin{tikzpicture}[overlay,gray!85,opacity=.85, line width=1.3pt]
	\draw [decorate, decoration={brace,amplitude=5pt}]
	($(exercise.south west)+(.05mm,0)$)
	-- ($(exercise.north west)+(.05mm,0)$);
	\draw [decorate, decoration={brace,amplitude=5pt,mirror}]
	($(exercise.south east)+(-.05mm,0)$)
	-- ($(exercise.north east)+(-.05mm,0)$);
	\draw
	($(exercise.south west)+(-5mm,2mm)$)
	-- +(-45mm,0);
	\draw
	($(exercise.south east)+(5mm,2mm)$)
	-- +(45mm,0);
	\end{tikzpicture}
	}

\renewcommand{\figurename}{图}

\newcommand\MySecSquare{%
	\leavevmode\hbox to 1.2ex{\hss\vrule height 1.3ex width 1.1ex depth -.2ex\hss}}


\definecolor{structurecolor}{RGB}{60,113,183}


\titleformat%
{\chapter}[hang]%
{\bfseries}{%
\begin{minipage}[t]{0.2\linewidth}
\vspace{0pt}% do not remove
\begin{tikzpicture}
\node[
outer sep=0pt,
text width=20mm,
minimum height=15mm,
fill=Blue4,
font=\color{white}\fontsize{53}{4}\selectfont,
align=center
] (num) {\thechapter};
\node[
outer sep=0pt,
inner sep=0pt,
anchor=south,
font=\color{Blue4}\Large\normalfont
] at ([yshift=3pt]num.north) {\textsc{C\,h\,a\,p\,t\,e\,r}};
\end{tikzpicture}
\end{minipage}%
}
{0pt}%
{%
\begin{minipage}[t]{.7\linewidth}%
    \vspace{2pt} % do not remove
    \color{Blue4}
    \rule{\linewidth}{2.5pt}\\\vskip -1.75\baselineskip%
    \rule{\linewidth}{.7pt}\vskip 5pt
    {\LARGE\raggedright\textsf{#1}}
\end{minipage}%
}


\titleformat{\section}
{\normalfont\Large\fd\color{Blue3}}{\MySecSquare\ \;\thesection}{1em}{#1}
\titleformat{name=\section,numberless}
{\normalfont\Large\fd}{\MySecSquare}{1em}{#1}


\titleformat{\subsection}
{\normalfont\large\fd\color{miunblue}}
{\tikz \fill[miunblue](0,0) circle (.15) ;\;\;\thesubsection}{1em}{#1}
\titleformat{name=\subsection,numberless}
{\normalfont\large\fd\color{miunblue}}
{\tikz \fill[miunblue](0,0) circle (.15) ;}{1em}{#1}
